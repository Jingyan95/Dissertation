\begin{acknowledgments}

Confucius said ``Is it not delightful to have friends coming from afar?'' I am glad that I received a few visits from my brother during my graduate study. We spent a lot of time on the phone talking about all the happy and sad things that happened in the world. He was the ``closest'' person in my family and my best friend. I couldn’t have made it this far without his moral support.

I am incredibly grateful to Louise who took me as her first student at a time when I was really unsure about the direction of my life. I imagined it would not be an obvious decision for most PIs given that I had no prior experience in the field. She always respected and supported the way I approach my research while deeply caring about my growth as a researcher. I am forever in her debt.  

I would like to express my gratitude to members of the Northeastern HEP community, especially Ela, from whom I learned about particle physics. Her enthusiastic and insightful lectures truly captured my heart and inspired me to pursue an adventure in particle physics.

Special thanks go to my parents and grandparents, who never forgot to send me their encouraging words when I needed the most. Despite not receiving much formal education themselves, they were always curious about my research, which matters a lot to me.

With COVID-19 and constantly moving in foreign countries, it was hardly easy for me to have a ``normal'' graduate school experience. In this regard, I would like to thank Bingran who helped me in many ways when I first relocated to CERN. I would also like to thank Amrutha, Meng, Nick, and Yixiao who were always ready to get together for my silly activities. 

Thank you to Emily who has made invaluable contributions to the second analysis presented in this thesis. I have always appreciated her dedication and bright ideas. It was a wonderful experience working with her even though we have never met in real life.

\end{acknowledgments}