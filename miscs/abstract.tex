\begin{abstract}
A search for charged-lepton flavor violation is performed in the top quark sector (t) through both top quark production and decay signal processes. The data were collected by the CMS experiment from proton-proton collisions at a center-of-mass energy of 13 TeV and correspond to an integrated luminosity of 138 fb$^{-1}$. The selected events are required to contain one opposite-sign electron-muon pair, a third charged lepton (electron or muon), at least one jet of which at most one is associated with a bottom quark. The analysis utilizes boosted decision trees to distinguish background processes from a possible signal, exploiting differences in the kinematics of the final state particles. The data are found to be consistent with the standard model expectation. Upper limits at 95\% confidence level are placed in the context of effective field theory on the Wilson coefficients, which range between 0.024--0.424$\TeV^{-2}$ depending on the flavor of the associated light quark and the Lorentz structure of the interaction. Upper limits on the Wilson coefficients are converted to upper limits on branching fractions involving up (charm) quarks, $\tto{u}$ ($\tto{c}$), of $0.032 \times 10^{-6}$ ($0.498 \times 10^{-6}$), $0.022 \times 10^{-6}$ ($0.369 \times 10^{-6}$), and $0.012 \times 10^{-6}$ ($0.216 \times 10^{-6}$) for tensor, vector, and scalar interactions, respectively.

Preliminary results from a second search for charged-lepton flavor violation is also presented in this thesis. Using only simulated samples, this search extends the scope of the first search by introducing hadronic tau leptons. Events selected in this search contain exactly two charged leptons (electron or muon) and one hadronic tau lepton. This search simultaneously looks for all three charged-lepton flavor mixing modes (i.e. e$\upmu$, e$\uptau$, and $\upmu\uptau$) through top quark production and decay processes. The second search complements the results from first search by providing strong sensitivity in $\upmu\uptau$ and e$\uptau$ flavor mixing modes.
\end{abstract}