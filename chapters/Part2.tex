\part{Experimental Apparatus}
\label{Part2}
The \autoref{Part2} of this thesis gives a brief description of the experimental apparatus that provides the physical environment and data collection for analyses described in \autoref{Part3} and \autoref{Part4}. Questions raised in \autoref{sec:Flavor} require the refinement of our knowledge of nature at its smallest scale. The \ac{LHC} is best prepared to help us achieve this goal as it is the most powerful particle physics accelerator in the world, colliding protons at a center-of-mass energy of 13.6 TeV in 2023. The \ac{CMS} detector is one of several detectors that is capable of recording data under the harsh physical environment created by the \ac{LHC}. The \autoref{Part2} is organized as follows. \autoref{chap:LHC} discusses the \ac{LHC} and its surrounding \ac{CERN} accelerator complex. An overview of the \ac{CMS} detector is given in \autoref{chap:CMS}. Details on event reconstruction in \ac{CMS} are given in \autoref{chap:Event}. I personally contributed to the \ac{CMS} operations and Phase-2 Upgrade, which are discussed in \autoref{chap:Ops} and \autoref{chap:Upgrade}, respectively. Materials presented in \autoref{chap:LHC}-\autoref{chap:Event} are borrowed from various publications and public documents, to which I made no direct contributions. Except where noted, materials (i.e. figures and tables) presented in \autoref{chap:Upgrade} are prepared by myself.
\chapter{The Large Hadron Collider}
\label{chap:LHC}

The \ac{LHC}~\cite{Evans:2008zzb} is a circular particle physics accelerator located near Geneva, Switzerland. It collides protons at four interaction points, which correspond to the four major experiments hosted by the \ac{LHC}: the \ac{ALICE}~\cite{ALICE:2008ngc}, \ac{ATLAS}~\cite{ATLAS:2008xda}, \ac{CMS}~\cite{CMS:2008xjf}, and \ac{LHCb}~\cite{LHCb:2008vvz} experiments. Accelerating proton beams to TeV-level requires a chain of acceleration stages before they are energetic enough for the final injection into the \ac{LHC} ring. These stages of acceleration and the whole \ac{CERN} accelerator complex are discussed in \autoref{sec:CERN}. The number of collision events delivered by the \ac{LHC} is measured in units called ``luminosity''. The definition of luminosity is discussed in \autoref{sec:Lumi}. The long term schedule for the \ac{LHC} is discussed in \autoref{sec:Plan}.

\section{The CERN Accelerator Complex}
\label{sec:CERN}

Installed in a 27 km circular tunnel previously used for the \ac{LEP} collider~\cite{203828}, the \ac{LHC} is the largest and most powerful particle accelerator ever existed. The primary objective of the \ac{LHC} is to deliver high intensity and high energy proton collisions, allowing physicists to study the laws of nature at the most fundamental scale. This objective is achieved through a complex system of accelerators, which rapidly accelerates protons to the target energy in a multistage process, maintains the proton energy, focuses, and collides them precisely at the designated locations. The full system is illustrated in Figure~\ref{fig:LHC}.

\begin{figure}[tbh!]
 \begin{center}
 \begin{tabular}{c}
 \includegraphics[width=\textwidth]{figures/Part2/LHC/CERN}
 \end{tabular}
 \caption{Layout of the \ac{CERN} accelerator complex, adapted from~\cite{CERN:2022}.}
 \label{fig:LHC}
 \end{center}
\end{figure}

Until 2018, an older linear accelerator (LINAC 2)~\cite{Boltezar:1979ba} was used to initially accelerate protons to 50 MeV. After 2018, negative hydrogen ions (H$^{-}$) are accelerated by the Linear Accelerator 4 (LINAC 4)~\cite{Vretenar:2020quc} to 160 MeV using cylindrical conductors charged by radiofrequency cavities. Quadrupole magnets are placed along the accelerator to keep the beam focused.  The hydrogen ions are then stripped of their two electrons during injection into the Proton Synchrotron Booster (BOOSTER)~\cite{Reich:1969fw}, which is a circular accelerator that boosts protons to 2 GeV. Protons are then injected into the Proton Synchrotron (PS)~\cite{PS:1953}, which was \ac{CERN}'s first synchrotron. The PS utilizes alternating-gradient focusing, a principle developed by Brookhaven physicists~\cite{Courant:1952rn,PhysRev.88.1197}, and accelerates protons to 25 GeV before injecting them into the second largest machine in the accelerator complex called the \ac{SPS}~\cite{Adams:1970en}. The \ac{SPS} is a 7 km circular accelerator that uses room-temperature dipole magnets to bend the protons. Protons are accelerated by the \ac{SPS} to an energy of 450 GeV before entering the final accelerator ring -- the \ac{LHC}. The \ac{LHC} uses super-conducting dipole magnets up to 8.4 T to bend protons and ultimately accelerate them to 6.8 TeV during Run-3. Quadrupole magnets are placed at four collision points to focus the proton beams, which eventually collide at the \ac{IP} of each detector. 

\section{Luminosity}
\label{sec:Lumi}

The total number of events of a given process is given by 

\begin{equation}
N=\int\mathcal{L}\sigma dt,
\end{equation}

where $\sigma$ is the cross-section of the process and $\mathcal{L}$ is known as the instantaneous luminosity that can be written in a simplified form

\begin{equation}
\mathcal{L}=\frac{N^2f}{A},
\end{equation}

where $N$ is the total number of protons in each beam, $f$ is the frequency of the beam crossing, and $A$ is the cross-sectional area of the beam crossing.

The \ac{LHC} is designed to deliver an instantaneous luminosity $\mathcal{L}=10^{34}\textsf{cm}^{-2}~\textsf{s}^{-1}$ which corresponds to an event rate of 1 billion collisions per second, assuming the inelastic cross-section $\sigma^{\textsf{pp}}_{\textsf{in}}~=$ 100 mb. The delivered integrated luminosity by year of data taking is shown in Figure~\ref{fig:twikilumi} (left).

The peak instantaneous luminosity reached $2\times10^{34}~\textsf{cm}^{-2}\textsf{s}^{-1}$ in 2018, which is factor of two larger than the design value of the \ac{LHC}. High instantaneous luminosity means more collision events are delivered by the \ac{LHC}, but it also brings a side effect -- multiple interactions per crossing, also known as \ac{PU} . The average number of \ac{PU} increased from 27 in 2016 to 52 in 2023, creating challenges for data-taking and event reconstruction. The \ac{PU} profile for each year of data taking is shown in Figure~\ref{fig:twikilumi} (right).

\begin{figure}[tbh!]
 \begin{center}
 \begin{tabular}{cc}
 \includegraphics[width=0.48\textwidth]{figures/Part2/LHC/twikilumi}&
 \includegraphics[width=0.48\textwidth]{figures/Part2/LHC/twikipu}\\
 \end{tabular}
 \caption{Delivered integrated luminosity versus time (left) and recorded luminosity versus mean number of interactions per crossing (left), adapted from~\cite{twiki:lumi}.}
 \label{fig:twikilumi}
 \end{center}
\end{figure} 

\section{LHC Long Term Schedule}
\label{sec:Plan}

The \ac{LS} 2 lasted for over 3 three years until the \ac{LHC} resumed data taking in mid 2022. The on-going run of the \ac{LHC}, known as Run-3, is expected to end in 2025. Between 2026 and 2028 is a period known as the \ac{LS} 3, when major upgrades of the \ac{LHC} and the hosted experiments will take place. A new era of the \ac{LHC}, known as the \ac{HL-LHC}~\cite{Apollinari:2017lan} will arrive in 2029, after which the instantaneous luminosity will gradually increase by up to a factor $7.5$ more than the designed value. The \ac{HL-LHC} is expected to be operational for more than 10 years until the 2040s. A summary of the \ac{LHC} future schedule is shown in Figure~\ref{fig:Schedule}.

\begin{figure}[tbh!]
 \begin{center}
 \begin{tabular}{c}
 \includegraphics[width=\textwidth]{figures/Part2/LHC/Schedule}
 \end{tabular}
 \caption{Future long term schedule of the \ac{LHC}, adapted from~\cite{LHC:plan} in November 2023.}
 \label{fig:Schedule}
 \end{center}
\end{figure}
\chapter{The Compact Muon Solenoid Detector}
\label{chap:CMS}

The \ac{CMS} detector~\cite{CMS:2008xjf} is one of the two general-purpose detectors involved in the discovery of the Higgs boson in 2012~\cite{ATLAS:2012yve,CMS:2012qbp}. It is located around 100 meters underground near the French town of Cessy. The full detector weights over 14 thousand tones, and is roughly cylindrically symmetric with a length and diameter of 21 and 15 meters, respectively. It consists of several layers of subsystems, as illustrated in Figure~\ref{fig:CMS}.

\begin{figure}[tbh!]
 \begin{center}
 \begin{tabular}{c}
 \includegraphics[width=0.8\textwidth]{figures/Part2/CMS/cms}
 \end{tabular}
 \caption{A sectional view of the \ac{CMS} detector, adapted from~\cite{Sakuma:2013jqa}.}
 \label{fig:CMS}
 \end{center}
\end{figure}

Brief descriptions of these subsystems are given in \autoref{sec:TK}-\autoref{sec:TrigSys}. The coordinate system adopted by \ac{CMS} is introduced in \autoref{sec:Coord}.

\section{Coordinate System Used in the CMS Detector}
\label{sec:Coord}

As illustrated in Figure~\ref{fig:axis3D}, the coordinate system adopted by \ac{CMS} uses the nominal \ac{IP} as its origin, with the $x$-axis pointing radially inward towards the center of the \ac{LHC} ring, the $y$-axis pointing vertically upward towards the sky, and the z-axis pointing along the beam line towards west of the detector.

\begin{figure}[tbh!]
 \begin{center}
 \begin{tabular}{c}
 \includegraphics[width=0.8\textwidth]{figures/Part2/CMS/axis3D_CMS-004}
 \end{tabular}
 \caption{A sketch of the coordinate system adopted by \ac{CMS}, adapted from~\cite{tikz:3D}.}
 \label{fig:axis3D}
 \end{center}
\end{figure}

The $x$- and $y$-axis form the transverse plane as they are both orthogonal to the beam line (z-axis). The distance from the \ac{IP} in the transverse plane is defined as $r=\sqrt{x^2+y^2}$. Variables defined entirely in the transverse plane, such as $\pt$, \MET, and $\Ht$, are often indicated by a subscripted T. The azimuthal angle $\phi$ is measured from the positive $x$-axis and the polar angle $\theta$ is measured from the positive z-axis. Another variable $\eta$, known as pseudorapidity, is defined as

\begin{equation}
\eta=-\ln(\frac{\theta}{2}).
\end{equation}

It is preferred over $\theta$ mainly due to: i) particle production rate is roughly uniform in this variable, and ii) a difference in this variables, denoted by $\mathrm{\Delta}\eta$, is invariant under Lorentz boosts. The conversion between $\eta$ and $\theta$ is illustrated in Figure~\ref{fig:axis2D}. The $\mathrm{\Delta}\eta$ and the difference in azimuthal angles, denoted by $\mathrm{\Delta}\phi$, are used to define the distance parameter $\mathrm{\Delta}R$

\begin{equation}
\label{eq:DR}
\mathrm{\Delta}R=\sqrt{\mathrm{\Delta}\eta^2+\mathrm{\Delta}\phi^2}.
\end{equation}

\begin{figure}[tbh!]
 \begin{center}
 \begin{tabular}{c}
 \includegraphics[width=0.8\textwidth]{figures/Part2/CMS/axis2D_pseudorapidity-003}
 \end{tabular}
 \caption{Examples of the conversion between the polar angle $\theta$ and the pseudorapidity $\eta$, adapted from~\cite{tikz:2D}.}
 \label{fig:axis2D}
 \end{center}
\end{figure}

\section{The Tracking System}
\label{sec:TK}

The tracking system~\cite{CMS:1997tlf} is the innermost subsystem of the \ac{CMS} detector where the density of particles from the collisions is the highest. The full tracking system is based on silicon technology to cope with the high radiation condition and provide excellent spatial resolutions while maintaining a light material budget. The alterations of charged particle trajectories caused by detector materials is expected to be minimal. Hits from charged particles, such as electrons and muons, are measured by silicon sensors and used reconstruct particle trajectories, known as ``tracks''.  The curvature of tracks can be then used to determine the momentum of these final state particles. Tracks can also be used to reconstruct the \ac{PV}, which corresponds to the hard scattering in collision events, and the \ac{SV}, which corresponds to the decay of heavy particles, such as tau leptons. 

The tracking system gives a coverage of up to $|\eta|~<$ 2.5, and is comprised of a pixel detector and a strip detector, which are also collectively known as the tracker detector. When charged particles go through tracker layers, they knocked out electrons in detector materials. These electrons create electric pulses when they travel in electric fields, which are then amplified and detected in the readout electronics. The tracker aims to A sketch of the \ac{CMS} tracker created shortly before Run-1 of the \ac{LHC} is shown in Figure~\ref{fig:Tracker}.

\begin{figure}[tbh!]
 \begin{center}
 \begin{tabular}{c}
 \includegraphics[width=0.8\textwidth]{figures/Part2/CMS/Tracker}
 \end{tabular}
 \caption{Layout of one quadrant of the \ac{CMS} tracker in the $r-z$ plane, adapted from~\cite{CMS:2009dvy}. The strip tracker is shown in pink color while the original pixel detector with three barrel layers is shown in black color.}
 \label{fig:Tracker}
 \end{center}
\end{figure}

The pixel tracker is comprised of roughly 66 million silicon sensors~\cite{CMS:2014pgm}, and is divided into two subsystems: the Barrel Pixel (BPIX) and  Forward Pixel (FPIX). The BPIX is composed of three cylindrical layers with radii ranging from 44 mm to 102 mm. The FPIX is composed of two disks on each side of the forward region. To cope with the intensified \ac{LHC} conditions in Run-2 and improve the overall tracking performance, an upgraded version of the pixel detector, referred to as the Phase-1 pixel detector, was installed during the year-end technical stop between 2016 and 2017~\cite{CMSTrackerGroup:2020edz}. The Phase-1 pixel detector is comprised of roughly 124 million silicon sensors distributed over four BPIX layers and three FPIX disks on each side. Differences between the two pixel detectors are shown in Figure~\ref{fig:Pixel}.

\begin{figure}[tbh!]
 \begin{center}
 \begin{tabular}{c}
 \includegraphics[width=0.85\textwidth]{figures/Part2/CMS/Pixel}
 \end{tabular}
 \caption{A comparison between the original pixel detector and the upgraded pixel detector in the $r-z$ plane, adapted from~\cite{CMSTrackerGroup:2020edz}.}
 \label{fig:Pixel}
 \end{center}
\end{figure}

The strip tracker is much larger in size and is built around the pixel tracker. It is comprised of roughly 10 million silicon sensors and is divided into several subsystems: the tracker inner barrel (TIB), outer barrel (TOB), inner disks (TID), and endcaps (TEC). The TIB and TOB consist of four and six layers, respectively. The TID and TEC consist of three and nine disks, respectively.

\section{The Electromagnetic Calorimeter}
\label{sec:ECAL}

The \ac{ECAL}~\cite{CMS:1997ysd} is homogeneous calorimeter that encloses the tracker detector, and thus it is the second innermost subsystem of the \ac{CMS} detector. It helps determine the energy and position of electrons and photons through their electromagnetic interactions with detector materials. The \ac{ECAL} gives a coverage of up to $|\eta|~<$ 3.0, and is divided into three subsystems: the \ac{ECAL} barrel calorimeter (EB), preshower calorimeter (ES), and endcap calorimeter (EE). A sketch of the \ac{ECAL} layout is shown in Figure~\ref{fig:ECAL}.

\begin{figure}[tbh!]
 \begin{center}
 \begin{tabular}{c}
 \includegraphics[width=0.8\textwidth]{figures/Part2/CMS/ECAL}
 \end{tabular}
 \caption{Layout of one quadrant of the \ac{ECAL} in the $r-z$ plane, adapted from~\cite{Benaglia:2014aqa}.}
 \label{fig:ECAL}
 \end{center}
\end{figure}

Unlike the tracker, the \ac{ECAL} aims to stop electrons and photons entirely. It is composed of over 76 thousand lead tungstate (PbWO$_{4}$) crystals, which act as absorber and scintillator simultaneously. When high-energy electrons and photons travel through these crystals, they create electromagnetic showers of low-energy electrons and photons. Over 98 \% of the shower energy can be absorbed and converted to light through the scintillation process. The scintillation light is then amplified and detected by photodiodes. The energy of particles can be measured from the intensity of the scintillation light.

\section{The Hadronic Calorimeter}
\label{sec:HCAL}

The \ac{HCAL}~\cite{CMS:1997xji} is a sampling calorimeter placed outside of the \ac{ECAL}. It measures the energy of hadrons through their strong interactions with detector materials. It also plays a crucial role in the measurement of the total energy in collision events, which allows for the determination of the \ac{MET}. The \ac{HCAL} gives a coverage of up to $|\eta|~<$ 5.2, and is divided into four subsystems: the \ac{HCAL} barrel calorimeter (HB), endcap calorimeter (HE), outer barrel calorimeter (HO), and forward calorimeter (HF). A sketch of the \ac{HCAL} layout is shown in Figure~\ref{fig:HCAL}.

\begin{figure}[tbh!]
 \begin{center}
 \begin{tabular}{c}
 \includegraphics[width=0.9\textwidth]{figures/Part2/CMS/HCAL}
 \end{tabular}
 \caption{Layout of one quadrant of the \ac{HCAL} in the $r-z$ plane, adapted from~\cite{CMS:2009nwd}.}
 \label{fig:HCAL}
 \end{center}
\end{figure}

Similar to the \ac{ECAL}, the \ac{HCAL} aims to stop hadrons entirely as their energy are minimally affected by the tracker and \ac{ECAL}. Unlike the \ac{ECAL}, which is homogeneous in its detector materials, the \ac{HCAL} uses alternating layers of absorbers and scintillators made of different materials and hence it is classified as a sampling calorimeter. Brass absorber plates are placed between plastic scintillators plates in the HB and HE, which provides coverages of $|\eta|~<$ 1.3 and 1.3 $<~|\eta|~<$ 3.0, respectively. The HO is placed outside of the magnetic coil to provide additional materials in barrel. It uses materials from the \ac{CMS} magnet as absorbers. The HF is placed outside of the detector around the beam line. It provides coverage of forward jets and uses steel as absorbing materials. 

\section{The Superconducting Magnet}
\label{sec:Magnet}

The superconducting magnet~\cite{CMS:1997moj} is the central feature of the \ac{CMS} detector. It has a diameter of roughly 6 meters and full encloses the tracker, \ac{ECAL} and the \ac{HCAL} HB and HE. The magnet consists of two main parts: the steel return yoke and the superconducting solenoid. The yoke weights more than 10 thousand tons and its main role is to improve the homogeneity of the magnetic filed inside the tracker and return the magnetic flux to the solenoid. The superconducting solenoid is enclosed in the yoke and it produces a uniform magnetic field of $B~=~3.8$ T inside the tracker volume. The paths of charged particles are curved by this magnetic field so that their momenta can be inferred from the curvatures of the trajectories according to the equation

\begin{equation}
\pt = |q|B\rho,
\end{equation}

where $q$ is the charge of the particle, $B$ is the magnetic field in the z direction, and $\rho$ is the radius of the curvature. A map of  the magnetic field generated by the solenoid is shown in Figure~\ref{fig:Magnet}. The solenoid operates at a current of over 18 thousand A and a temperature of 4.2 K (-268.95 $^\circ$C). It is cooled by the liquid helium.

\begin{figure}[tbh!]
 \begin{center}
 \begin{tabular}{c}
 \includegraphics[width=0.8\textwidth]{figures/Part2/CMS/Magnet}
 \end{tabular}
 \caption{Predicted values of $|B|$ (left) and field lines (right) on a longitudinal section of the \ac{CMS} detector, adapted from~\cite{CMS:2009moq}.}
 \label{fig:Magnet}
 \end{center}
\end{figure}

\section{The Muon System}
\label{sec:MuonSys}

The muon system~\cite{CMS:1997iti} is located outside of the solenoid and embedded into the return yoke. It is the outermost and largest subsystem of the \ac{CMS} detector, consisting of several subsystems that give an overall coverage of up to $|\eta|~<$ 2.4. All these subsystems all based on gas-ionization technology: the \ac{DT} and \ac{RPC} together make up the barrel region of the muon system, and the endcap muon system consists of \ac{RPC} and  \ac{CSC}. The \ac{GEM}~\cite{Colaleo:2015vsq} is the latest addition to the muon system. It complements \ac{CSC} in the forward region. A sketch of the muon system layout is shown in Figure~\ref{fig:Muon}.

\begin{figure}[tbh!]
 \begin{center}
 \begin{tabular}{c}
 \includegraphics[width=1\textwidth]{figures/Part2/CMS/Muon}
 \end{tabular}
 \caption{Layout of one quadrant of the muon system in the $r-z$ plane, adapted from~\cite{Colaleo:2015vsq}.}
 \label{fig:Muon}
 \end{center}
\end{figure}

Muons produced in collision events retain most of their momenta when they penetrate the tracker and calorimeters, which allows for the precise determination of their trajectories by the muon system. The barrel region ($|\eta|~<$ 1.2) is occupied by four stations of \acp{DT}, which consists of charged wires and is filled with gas. The \ac{DT} is chosen for this region because the event rate is lower in this region and magnetic field is weaker but more uniform relative to the forward region. When muons pass through the \ac{DT} chambers, they knock out electrons from the gas atoms, which then move towards the positively charged wires due to its electric field. The charged wires are placed perpendicular to each other so that the $x$ and $y$ coordinates of the muon positions can be determined. 

In region (0.9 $<~|\eta|~<$ 2.4) where event rate is higher, four stations of the \ac{CSC} chambers are positioned on each side. Similar to the detection mechanism in the \ac{DT}, muons knock out electrons from gas atoms when they pass through the \ac{CSC} chambers. Unlike the \ac{DT}, which solely relies on positively charged wires, the \ac{CSC} uses positively charged wires and negatively charged strips positioned perpendicular to each other. When electrons move to the wires, they create an avalanche of electrons. At the same time, a signal in the strips will be created by the ionized gas atoms. These two signals together allows for the determination of the $x$ and $y$ coordinates of the muon positions. 

Complementary to the \ac{DT} and \ac{CSC}, the \ac{RPC} chambers are positioned in both barrel and endcap region ($|\eta|~<$ 1.9). They provide coarser position resolution but fast response and good time resolution, which is useful in the muon trigger. The \ac{GEM} will extend the coverage of the muon system to $|\eta|~<$ 2.8 and is expected to be fully operational before the start of the Run-4. 

\section{The Trigger System}
\label{sec:TrigSys}
 
The \ac{LHC} collides proton bunches every 25 ns, which corresponds to an event rate of 40 MHz. It is beyond the hardware limit to record every events offline. Moreover, majority of the events involve only soft \ac{QCD} processes that are of little interests to particle physicists. The \ac{CMS} trigger system~\cite{CMS:2016ngn} is designed to reduce the data volume to a feasible level and select events that are interesting to the physics programs at the \ac{CMS}. It consists of two layers: the \ac{L1} trigger and \ac{HLT}.

The \ac{L1} trigger is first layer of the trigger system and is based on custom-designed hardware.

\begin{figure}[tbh!]
 \begin{center}
 \begin{tabular}{c}
 \includegraphics[width=1\textwidth]{figures/Part2/CMS/L1T}
 \end{tabular}
 \caption{Diagram of the \ac{L1} trigger system during Run-2 of the \ac{LHC}, adapted from~\cite{CMS:2020cmk}.}
 \label{fig:L1T}
 \end{center}
\end{figure}

The \ac{HLT} is the second layer of the trigger system and is based on a software-based algorithms that run on computer farms. 
\chapter{Event Reconstruction in the CMS detector}
\label{chap:Event}

\section{Electron and Photon}
\label{sec:Electron}

\section{Muon}
\label{sec:Muon}

\section{Tau}
\label{sec:Tau}

\section{Jet}
\label{sec:Jet}

\section{Energy Sum}
\label{sec:MET}
\chapter{The Run-3 Operations of the CMS detector}
\label{chap:Ops}

The \ac{CMS} detector resumed data-taking in July 2022, following the start of the Run-3 of the \ac{LHC}. When compared to Run-2, the center of mass energy of the proton beams increases from 13 TeV to 13.6 TeV in Run 3. At the same time, the peak instantaneous luminosity is kept at the same or higher level as the 2018 data-taking year ($2\times10^{34}~\textsf{cm}^{-2}\textsf{s}^{-1}$), as illustrated in Figure~\ref{fig:peak}.

\begin{figure}[tbh!]
 \begin{center}
 \begin{tabular}{c}
 \includegraphics[width=\textwidth]{figures/Part2/Operation/peak_lumi_pp}
 \end{tabular}
 \caption{Peak luminosity versus day delivered to \ac{CMS} during stable beams and for proton-proton collisions, adapted from~\cite{twiki:lumi}.}
 \label{fig:peak}
 \end{center}
\end{figure}

Operations of the \ac{CMS} detector are coordinated by the \ac{CMS} Run Coordination, which is introduced in \autoref{sec:RC}. The \ac{CMS} control room is the commanding center of the \ac{CMS} operations, which is staffed 24$\times$7 during the active data-taking period. Personnel who monitor and operate the \ac{CMS} detector from the control room are referred to as the ``central shift crew``, which is discussed further in \autoref{sec:ControlRoom}. In the context of detector operations, the principal contacts of the subsystems of the \ac{CMS} detector are known as the \ac{DOC} experts, or simply the \acp{DOC}. Core duties of one of the \acp{DOC}, the tracker \ac{DOC}, are described in \autoref{sec:DOC}.

\section{The CMS Run Coordination}
\label{sec:RC}

The \ac{CMS} Run Coordination is nominally headed by two Run Coordinators and one deputy Run Coordinator whose mandate is to ensure the successful running of \ac{CMS}. The Run Coordinators oversee all operations activities at the \ac{CMS} and are nominally appointed for a two-year term. The Run Coordinators work closely with the Technical Coordination, the \ac{LHC} team, and the subsystem operations teams to draft the long-term strategic goals for the central operations, as well as commissioning efforts in subsystems. These strategic goals are helped achieved by the \acp{RFM} who serve as the liaison between the Run Coordination and the central shift crew, as illustrated in Figure~\ref{fig:RC}.

\begin{figure}[tbh!]
 \begin{center}
 \begin{tabular}{c}
 \includegraphics[width=0.9\textwidth]{figures/Part2/Operation/RC}
 \end{tabular}
 \caption{Main communication paths between various personnel within the \ac{CMS} Run organization. The \ac{SL}, technical, \ac{DAQ}, \ac{DQM}, and the trigger shifters are required to be present at the \ac{CMS} control room 24$\times$7. The \acp{RFM} and the \acp{DOC} are nominally present in the control room during working hours. The Run Coordinators, subsystem experts, and offline shifters are not required to be at the control room although they often do.}
 \label{fig:RC}
 \end{center}
\end{figure}

The \ac{RFM} team is nominally appointed for a two-week term and typically consists of two members who have extended experiences in various roles of operations, particularly the \ac{SL}. The \acp{RFM}, together with the Run Coordinators, organize the \ac{CMS} daily run meetings in the morning every weekday to collect the feedback \& requests from the subsystem \acp{DOC} and set the daily run plan. The \acp{RFM} also facilitates the \ac{SL} in implementing these plans as the \ac{SL}, like the rest of the central shift crew, nominally does not attend the daily run meeting.

\section{Central Shift Crew}
\label{sec:ControlRoom}

The \ac{CMS} detector is considered to be ``running'' when all high-voltage channels are switched on and taking data. During this period, operations of various subsystems of the \ac{CMS} detector are controlled by the central shift crew, which consists of five members: the \ac{SL}, technical, \ac{DAQ}, \ac{DQM}, and trigger shifters, as illustrated in Figure~\ref{fig:ControlRoom}. 

\begin{figure}[tbh!]
 \begin{center}
 \begin{tabular}{c}
 \includegraphics[width=0.4\textwidth]{figures/Part2/Operation/ControlRoom}
 \end{tabular}
 \caption{A sketch of the layout of the \ac{CMS} control room. The area where the event display and the safety panel are located is nominally not designated for any personnel. The \ac{PPS}~\cite{CMS:2014sdw} and \ac{BRIL}~\cite{CMS:2008xjf} groups have designated working space in the control room although these groups do not maintain a 24$\times$7 presence in the area.}
 \label{fig:ControlRoom}
 \end{center}
\end{figure}

In order to maximize the luminosity output and data-taking efficiency, the \ac{LHC} normally takes no break at night once it starts producing high-intensity collisions. The \ac{CMS} detector is therefore kept on 24$\times$7 once the \ac{LHC} is operational, and members of the \ac{CMS} Collaboration take eight-hour shifts in relay from the control room. The running period when all subsystems are included in the data taking is considered to be a ``global run''. The data collected from global runs are then screened and certified by the \ac{CMS} \ac{PPD} group for physics analyses. 

The \ac{SL} is nominally considered to be the leader of the central shift crew, whose primary duty is to ensure the successful execution of the daily run plan. The \ac{SL} coordinates all activities in the \ac{CMS} control room and communicates with the \acp{RFM}, \ac{CMS} Run Coordination, as well as the \ac{CCC} about the operations. The \ac{SL} is also simultaneously the \ac{SLIMOS}, who is in charge of the safety during the operation, together with the technical shifter.

The technical shifter is responsible for all things related to the \ac{DCS} and \ac{DSS}. The technical shifter is often the ``first responder'' when problems arise in the \ac{CMS} detector or the surrounding infrastructures during the operation. The technical shifter coordinates the responses to these problems with the \acp{DOC}, \ac{CMS} Technical Coordination, as well as \ac{CERN} technical team. The technical shifter is also co-responsible for safety during the operation as safety duties are nominally delegated from the \ac{SL} to the technical shifter. These duties include: monitoring and responding to the \ac{DSS} alerts, overseeing the underground access and the usage of all safety equipment in the control room, and performing safety tours in the surface as well as the underground area. 

The core duty of the \ac{DAQ} shifter is to ensure smooth and efficient data taking, where the efficiency is roughly measured as the ratio of the recorded luminosity and the delivered luminosity. When a subsystem \ac{DAQ} runs into problems it can stop a global run and block the whole system entirely from running again, resulting in the so-called ``downtime'', which undermines the taking data taking efficiency. The \ac{DAQ} shifter is in communication with the \ac{SL} and \acp{DOC} about the readiness of all subsystems before initializing a global run, and is heavily involved in troubleshooting when subsystems are uncooperative in global runs. 

The \ac{DQM} shifter is responsible for the quality of the data taken by the \ac{CMS} detector. The \ac{DQM} shifter is nominally the first person to spot problems (related to data quality) in the running detector and is trained to do so by familiarizing him- or herself with the ``normal'' as well as the faulty patterns of the data collected in all subsystems. Traditionally \ac{DQM} shifters attend shifts in person from the \ac{CMS} control room just like the \ac{SL}, technical, and \ac{DAQ} shifters. In recent years, especially since the COVID-19 outbreak, more flexibility has been given to the \ac{DQM} shifters, who can now choose to work from one of the \ac{CMS} Remote Operations Centers or home.

The \ac{L1} and \ac{HLT} rates during the data taking are monitored by the trigger shifter, who along with the \ac{SL} determines the appropriate prescale column based on the real-time \ac{L1} rate. The trigger shifter makes sure all trigger subsystems are running correctly and he or she is in communication with the \ac{L1} and \ac{HLT} \acp{DOC} when troubleshooting is needed. Similar to \ac{DQM} shifters, trigger shifters have the option to work remotely provided that they have done in-person shifts more than a few times and are sufficiently familiar with the procedure. 

\section{Tracker Detector On-call Expert}
\label{sec:DOC}

Subsystems like the tracker do not maintain a 24$\times$7 presence at the \ac{CMS} control room. Instead, their shifters, known as \acp{DOC}, are appointed for one week and nominally only join the central shift crew in the control room during working hours. After the working hours, \acp{DOC} remain accessible by phone and they are ready to go to the control room at any time should the situation require. 

The term ``tracker \ac{DOC}'' usually refers to the strip tracker \ac{DOC} while the pixel tracker has its own \ac{DOC} known as the ``pixel \ac{DOC}''. The tracker \ac{DOC} is the main point of contact for the strip tracker during his or her mandate, which typically lasts for one week. On behalf of the strip tracker operations team, the tracker \ac{DOC} reports the status of the strip tracker at the \ac{CMS} daily run meeting. He or she also monitors the state of the strip tracker in all aspects (e.g. power, \ac{DAQ}) and coordinates the daily activities with tracker detector experts and the tracker offline shift crew whose primary duty is the certification of the data collected by the tracker. 

A stable and safe operation of the strip tracker requires both well-trained tracker \acp{DOC} as well as a modern \ac{DCS}. Built on top of the industrial product “WinCC”, the \ac{CMS} \ac{TCS}~\cite{Shah:2009zz,Karimeh:2020tzx} is designed to monitor the environmental conditions and safely operate the detector. As part of the \ac{TCS} software, a \ac{FSM} toolkit is introduced. It is a powerful tool that assists operators in their daily jobs. It groups the power, cooling, dry gas, and monitoring systems defined in the four \ac{TCS} projects in one hierarchical tree. The global state of the detector is continuously evaluated and made visible from the root Tracker \ac{FSM} node giving critical information to the detector operator, as illustrated in Figure~\ref{fig:DCS}.

\begin{figure}[tbh!]
 \begin{center}
 \begin{tabular}{c}
 \includegraphics[width=0.8\textwidth]{figures/Part2/Operation/TrackerDCS}
 \end{tabular}
 \caption{Main panel of the tracker \ac{FSM}, screenshotted in October 2022 during the Run-3 data taking.}
 \label{fig:DCS}
 \end{center}
\end{figure}
\chapter{The Phase-2 Upgrade of the CMS Detector}
\label{chap:Upgrade}

Planned to start in 2029, the \ac{HL-LHC}~\cite{Apollinari:2017lan} will reach a peak instantaneous luminosity of up to $7.5\times10^{34}$cm$^{-2}$s$^{-1}$, as illustrated in Figure~\ref{fig:Lumi}. The increased luminosity will open up the opportunities for ambitious physics programs including precision \ac{SM} measurements and searches for physics \ac{BSM}. To fully exploit the physics potential offered by the \ac{HL-LHC} datasets and overcome the challenging operational conditions, such as intense radiation and up to 200 \ac{PU} per event, the \ac{CMS} detector will undergo substantial upgrades during the \ac{LS} 3, known as the Phase-2 Upgrade~\cite{Contardo:2015bmq}.  

\begin{figure}[tbh!]
 \begin{center}
 \begin{tabular}{c}
 \includegraphics[width=0.8\textwidth]{figures/Part2/Upgrade/Lumi}
 \end{tabular}
 \caption{The peak and intergraded luminosity expected to be delivered by the \ac{HL-LHC}, taken from~\cite{LHC:plan} in November 2023. The left-hand $y$-axis shows the scale of the peak instantaneous luminosity, which is itself represented with red dots. The right-hand $y$-axis shows the scale of the intergraded luminosity. The two solid lines represent the intergraded luminosity under two \ac{YETS} scenarios.}
 \label{fig:Lumi}
 \end{center}
\end{figure}

An overview of the Phase-2 Upgrade is given in \autoref{sec:Overview}. Among various systems upgrades, the upgrade of the Outer Tracker is more relevant to this thesis, which is described in \autoref{sec:OT}. The Outer Tracker upgrade will enable tracking at the \ac{L1} trigger, which is discussed in \autoref{sec:Algo}. The tracking information can be combined with the calorimeter responses to build electron candidates at the \ac{L1} trigger, which is discussed in \autoref{sec:L1Ele}.

\section{Overview of the Upgrade}
\label{sec:Overview}

A new silicon tracker~\cite{CMS:2017lum} will replace the current tracker for the Phase-2. The Phase-2 tracker is divided into two subsystems: a pixel detector known as the Inner Tracker and the Outer Tracker composed of strip and macro-pixel sensors. The Phase-2 Tracker will provide efficient tracking up to $|\eta|~<$ 4 because of the extended coverage of the Inner Tracker. The Phase-2 tracker is much lighter with improved radiation hardness while enjoying a reduced material budget in the tracking volume. The granularity of Phase-2 tracker will be increased by roughly a factor of 4, leading to a much better charged-particle $\pt$ resolution. More importantly, the Phase-2 Outer Tracker is specially designed to be capable of delivering data to the \ac{L1} trigger, which is further discussed in \autoref{sec:OT}.

The \ac{L1} trigger~\cite{Zabi:2020gjd}

The electronics of the \ac{ECAL} Barrel Calorimeter will be replaced to accomodate the \ac{L1} trigger requirements on latency and rate. The upgraded electronics will enable the Phase-2 \ac{L1} trigger to exploit the information from single crystals.

The \ac{ECAL} and \ac{HCAL} Endcap Calorimeters will be replaced by a new endcap calorimeter known as the \ac{HGCAL}~\cite{CMS:2017jpq}. The \ac{HGCAL}

The \ac{MTD}~\cite{Butler:2019rpu}

The Muon system~\cite{Hebbeker:2017bix}

The \ac{HLT} ~\cite{HLT:Upgrade}

Upgrades of the \ac{BRIL} system is also planned and is documented in details in Ref.~\cite{Beam:Upgrade}.

\section{The Outer Tracker Upgrade}
\label{sec:OT}

The trigger primitives produced by the Outer Tracker, known as ``stubs'', form input to the track finding algorithm. The initial stage of the algorithm involves correlating stubs from two adjacent layers to form ``tracklets''. The tracklets are then used as seeds for extrapolating to other layers and disks to match with additional stubs. A \ac{KF} takes all potential tracks from combinations of tracklet and stubs, and produces the best track fit.

\begin{figure}[tbh!]
 \begin{center}
  \begin{tabular}{c}
   \centering\includegraphics[width=0.9\linewidth]{figures/Part2/Upgrade/TrackerGeo}
  \end{tabular}
  \caption{(Layout of one quadrant of the Phase-2 \ac{CMS} tracker in the $r-z$ plane, generated by the \ac{CMSSW}~\cite{cmssw}. The PS modules of the Outer Tracker are represented with blue lines while the 2S modules are represented with red lines. The inner tracker modules are represented with orange and green lines, which do not contribute to the \ac{L1} trigger.}
 \label{fig:TrackerGeo}
 \end{center}
\end{figure}

The trigger primitives produced by the Outer tracker, known as ``stubs'', form input to the track finding algorithm. The initial stage of the algorithm involves correlating stubs from two adjacent layers to form ``tracklets''. The tracklets are then used as seeds for extrapolating to other layers and disks to match with additional stubs. A \ac{KF} takes all potential tracks from combinations of tracklet and stubs, and produces the best track fit.

The trigger primitives produced by the Outer tracker, known as ``stubs'', form input to the track finding algorithm. The initial stage of the algorithm involves correlating stubs from two adjacent layers to form ``tracklets''. The tracklets are then used as seeds for extrapolating to other layers and disks to match with additional stubs. A \ac{KF} takes all potential tracks from combinations of tracklet and stubs, and produces the best track fit.

\begin{figure}[tbh!]
 \begin{center}
  \begin{tabular}{c}
   \centering\includegraphics[width=0.95\linewidth]{figures/Part2/Upgrade/Modules}
  \end{tabular}
  \caption{Illustrations of the 2S module (left) and PS module (right), adapted from~\cite{CMS:2017lum}. Shown are views of the assembled modules (top) and  sketches of the frontend hybrid folded assembly and connectivity (bottom).}
 \label{fig:Modules}
 \end{center}
\end{figure}

To maintain a low trigger threshold even with the harsh environment, the CMS experiment is planning to replace its entire Level-1 (L1) trigger system. One of the key goals of the L1 trigger upgrade is to incorporate tracking information of the charged particles. This goal also drives the design of the Outer Tracker at \ac{HL-LHC}, which utilizes the $\pt$ modules (See Figure \ref{fig:TrackerGeo}) to produce ``trigger primitive'' for the reconstruction of the trajectories of charged particles (known as ``tracks''). Each $\pt$ module consists of two closely spaced silicon sensors. By correlating hits from two layers, the $\pt$ modules are capable of providing $\pt$ discrimination at the front end. With current design, only hits over a threshold of 2 GeV will be read out. 

\begin{figure}[tbh!]
 \begin{center}
  \begin{tabular}{cc}
   \centering\includegraphics[width=0.9\linewidth]{figures/Part2/Upgrade/Stub}
  \end{tabular}
  \caption{(Left) Tracker geometry at the \ac{HL-LHC}. Outer Tracker consists of over 3000 $\pt$ modules. (Right) Cross section of the $\pt$ module.}
 \label{fig:Stub}
 \end{center}
\end{figure}

\section{Leve-1 Track Finder}
\label{sec:Algo}

The trigger primitives produced by the Outer Tracker, known as ``stubs'', form input to the track finding algorithm. The initial stage of the algorithm involves correlating stubs from two adjacent layers to form ``tracklets''. The tracklets are then used as seeds for extrapolating to other layers and disks to match with additional stubs. A \ac{KF} takes all potential tracks from combinations of tracklet and stubs, and produces the best track fit.

\begin{figure}[tbh!]
 \begin{center}
  \begin{tabular}{cc}
   \includegraphics[width=.45\linewidth]{figures/Part2/Upgrade/tracklet1} &
   \includegraphics[width=.45\linewidth]{figures/Part2/Upgrade/tracklet2} \\
   \includegraphics[width=.45\linewidth]{figures/Part2/Upgrade/tracklet3} &
   \includegraphics[width=.45\linewidth]{figures/Part2/Upgrade/tracklet4} \\
  \end{tabular}
  \caption{Illustration of different stages of the track finding algorithm: (a) constructing stubs, (b) forming tracklet by correlating two stubs and the beam spot (origin), (c) projecting to other layers and finding matches, and (d) fitting track parameters.}
 \label{fig:algorithm}
 \end{center}
\end{figure}

The baseline version of the algorithm requires at least four stubs and constrains the origin of the trajectories to the beam spot. An ``extended'' version of the algorithm is also in development, in which the beam spot constraint is relaxed. 

The track finding algorithm has demonstrated a robust performance (See Figure \ref{fig:trackingperformance}) in software simulation and is currently being tested on physical hardware. The project is progressing well and is on track for delivery before the start of the \ac{HL-LHC} in 2029. 

 \begin{figure}[tbh!]
 \begin{center}
  \begin{tabular}{cc}
   \includegraphics[width=.45\linewidth]{figures/Part2/Upgrade/L1TK_ttbar-pu200_eff_eta}&
   \includegraphics[width=.45\linewidth]{figures/Part2/Upgrade/L1TK_ttbar-pu200_resVsEta_z0}
  \end{tabular}
  \caption{(Left) Tracking efficiency vs particle $\eta$, measured in $\ttbar$ samples. (Center) Track $z_0$ resolution vs particle $\eta$. (Right) Electron tracking efficiency vs particle $\eta$.}
 \label{fig:trackingperformance}
 \end{center}
\end{figure} 

 \begin{figure}[tbh!]
 \begin{center}
  \begin{tabular}{cc}
   \includegraphics[width=.45\linewidth]{figures/Part2/Upgrade/L1TK_elec-pu0_eff_pt}&
   \includegraphics[width=.45\linewidth]{figures/Part2/Upgrade/L1TK_elec-pu0_eff_eta}
  \end{tabular}
  \caption{(Left) Tracking efficiency vs particle $\eta$, measured in $\ttbar$ samples. (Center) Track $z_0$ resolution vs particle $\eta$. (Right) Electron tracking efficiency vs particle $\eta$.}
 \label{fig:electronperformance}
 \end{center}
\end{figure} 

\section{Leve-1 Electron Trigger Algorithm}
\label{sec:L1Ele}

With the increase of the instantaneous luminosity, triggering on electrons will face unprecedented challenges as the data volume becomes too large to be recorded. The addition of tracking information provides a much-needed handle for the electron trigger. It enables precise track-cluster matching to lower the trigger rate. To take advantage of this new tool, a new electron trigger algorithm is developed: L1 tracks are propagated to the calorimeter surface to match with calorimeter clusters. An elliptical cut in the $\eta-\phi$ plane is applied and illustrated in Figure \ref{fig:electron} (left). Figure \ref{fig:electron} (center) demonstrates the small efficiency drop relative to the calorimeter-only algorithm. Figure \ref{fig:electron} (right) demonstrates the resulting sizable rate reduction.  
  
 \begin{figure}[tbh!]
 \begin{center}
  \begin{tabular}{ccc}
   \includegraphics[width=.45\linewidth]{figures/Part2/Upgrade/DR_barrel}&
   \includegraphics[width=.45\linewidth]{figures/Part2/Upgrade/DR_endcap}&
  \end{tabular}
  \caption{(Left) $\mathrm{\Delta}\eta$ vs $\mathrm{\Delta}\phi$ distances between calorimeter clusters and the closest L1 track. (Center) single electron efficiency as a function of the generated $\pt$. The efficiency drop is largely driven by electron track reconstruction which is also reflected in Figure~\ref{fig:trackingperformance}. (Right) trigger rate as a function of the cluster $\pt$.}
 \label{fig:electron}
 \end{center}
\end{figure}

When compared to the existing track-electron algorithm, this newly developed algorithm improved the electron identification efficiency by about 5$\%$ while reducing trigger rates by a factor of 2. The firmware implementation of this new algorithm is under development. Further improvement of this algorithm is possible with the ``extended'' tracking as well as a dedicated electron track quality classifier.

 \begin{figure}[tbh!]
 \begin{center}
  \begin{tabular}{ccc}
   \includegraphics[width=.45\linewidth]{figures/Part2/Upgrade/DR_barrel_new}&
   \includegraphics[width=.45\linewidth]{figures/Part2/Upgrade/DR_endcap_new}&
  \end{tabular}
  \caption{(Left) $\mathrm{\Delta}\eta$ vs $\mathrm{\Delta}\phi$ distances between calorimeter clusters and the closest L1 track. (Center) single electron efficiency as a function of the generated $\pt$. The efficiency drop is largely driven by electron track reconstruction which is also reflected in Figure~\ref{fig:trackingperformance}. (Right) trigger rate as a function of the cluster $\pt$.}
 \label{fig:DR_electron}
 \end{center}
\end{figure}

 \begin{figure}[tbh!]
 \begin{center}
  \begin{tabular}{ccc}
   \includegraphics[width=.45\linewidth]{figures/Part2/Upgrade/eff_barrel}&
   \includegraphics[width=.45\linewidth]{figures/Part2/Upgrade/eff_endcap}&
  \end{tabular}
  \caption{(Left) $\mathrm{\Delta}\eta$ vs $\mathrm{\Delta}\phi$ distances between calorimeter clusters and the closest L1 track. (Center) single electron efficiency as a function of the generated $\pt$. The efficiency drop is largely driven by electron track reconstruction which is also reflected in Figure~\ref{fig:trackingperformance}. (Right) trigger rate as a function of the cluster $\pt$.}
 \label{fig:eff_electron}
 \end{center}
\end{figure}

 \begin{figure}[tbh!]
 \begin{center}
  \begin{tabular}{ccc}
   \includegraphics[width=.45\linewidth]{figures/Part2/Upgrade/Rate_barrel}&
   \includegraphics[width=.45\linewidth]{figures/Part2/Upgrade/Rate_endcap}&
  \end{tabular}
  \caption{(Left) $\mathrm{\Delta}\eta$ vs $\mathrm{\Delta}\phi$ distances between calorimeter clusters and the closest L1 track. (Center) single electron efficiency as a function of the generated $\pt$. The efficiency drop is largely driven by electron track reconstruction which is also reflected in Figure~\ref{fig:trackingperformance}. (Right) trigger rate as a function of the cluster $\pt$.}
 \label{fig:rate_electron}
 \end{center}
\end{figure}
