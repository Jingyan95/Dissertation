\chapter{Electroweak Theory}
\label{chap:SM}

Developed in the 1960s, the Glashow–Weinberg–Salam theory of \ac{EW} interaction is regarded by many physicists as the cornerstone of the \ac{SM}. It successfully unified the weak interaction with electromagnetism and postulated the existence of several new particles, all of which were later confirmed by experiments. One of the most important aspects of this theory is the \ew~gauge symmetry, which is discussed in \autoref{sec:Gauge}. The \ew~ symmetry is spontaneously broken through the Higgs mechanism, which is discussed in \autoref{sec:Higgs}. Finally, flavor physics and its connection to the Yukawa interaction is discussed in \autoref{sec:Flavor}. 

\section{Gauge Theory}
\label{sec:Gauge}

\section{Higgs Mechanism}
\label{sec:Higgs}

\section{Flavor Sector}
\label{sec:Flavor}