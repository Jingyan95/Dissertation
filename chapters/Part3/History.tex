\chapter{Previous searches}
\label{chap:History}

\ac{CLFV} search involving top quarks is an active area of research at the \ac{LHC} experiments. So far, no significant excess over the \ac{SM} predictions has been reported, and the observations from the \ac{ATLAS} and \ac{CMS} experiments are both interpreted using the framework of \ac{EFT}. Two existing \ac{ATLAS} analyses are described in \autoref{sec:CLFV_ATLAS} and one existing \ac{CMS} analysis is described in \autoref{sec:CLFV_CMS}.
%%%%%%%%%%%%%%%%%%%%%%%%%%%%%%%%%%%%%%%%%%%
%%%%%%%%%%%%%%%%%%%%%%%%%%%%%%%%%%%%%%%%%%%

\section{ATLAS}
\label{sec:CLFV_ATLAS}

The flavor-violating $\emut{q}$ interactions were first studied by the \ac{ATLAS} Collaboration~\cite{ATLAS-CONF-2018-044} using data collected during 2015-2017 with an integrated luminosity of 79.8 fb$^{-1}$. In addition to three leptons, this analysis targets final states with two or more jets. Only the top quark decay signal mode is considered. Lorentz structures of dimension-6 operators are not probed separately. Discriminating variables, such as the $\pt$ of the leptons, are combined into a \ac{BDT}, which is used to interpret the observation. A representative Feynman diagram targeted by this analysis and the distributions of the \ac{BDT} discriminator are shown in Figure~\ref{fig:ATLAS_results1}.

\begin{figure}[tbh!]
 \begin{center}
 \begin{minipage}[b]{0.325\linewidth} 
  \includegraphics[width=\textwidth]{figures/TT} 
  \vspace{2em}
 \end{minipage}
 \hfill
 \begin{minipage}[b]{0.325\linewidth} 
  \includegraphics[width=\textwidth]{figures/Part3/History/ATLAS_results1}
 \end{minipage}
 \hfill
 \begin{minipage}[b]{0.325\linewidth} 
  \includegraphics[width=\textwidth]{figures/Part3/History/ATLAS_results2}
 \end{minipage}
 \caption{Representative Feynman diagram for the \ac{CLFV} top quark decay processes that are targeted by \cite{ATLAS-CONF-2018-044} (left). The \ac{CLFV} interaction vertex is shown as a solid red circle to indicate that it is not allowed in the \ac{SM}. The middle (right) histogram shows the distribution of the pre-fit (post-fit) \ac{BDT} discriminator targeting the \ac{CLFV} top quark decay. }
 \label{fig:ATLAS_results1}
 \end{center}
\end{figure}

Data is found to be compatible with the \ac{SM} predictions, and an upper limit on the branching fraction of $\mathcal{B}(\tto{q}$) $<$ 6.6 $\times$ 10$^{-6}$ is set at 95\% \ac{CL}~\cite{Read2002}. This result improves a previous bound established in an indirect search~\cite{Davidson:2015zza} by three orders of magnitude. 

The \ac{ATLAS} Collaboration also studied the $\upmu\uptau$tq interactions using data corresponds to 140 fb$^{-1}$~\cite{ATLAS-CONF-2023-001}. This analysis targets final states with two same-sign muons, one hadronic tau, and one or more jets. Both top quark production and decay signals are considered by this analysis. Operators with different Lorentz structures are considered separately. Representative Feynman diagrams are shown in Figure~\ref{fig:ATLAS_FD}. 

\begin{figure}[tbh!]
 \begin{center}
 \begin{tabular}{ccc}
  \includegraphics[width=0.31\textwidth]{figures/Part3/History/ATLAS_TT}&
  \includegraphics[width=0.33\textwidth]{figures/Part3/History/ATLAS_ST1}&
  \includegraphics[width=0.31\textwidth]{figures/Part3/History/ATLAS_ST2}\\
 \end{tabular}
 \caption{Representative Feynman diagrams for the signal processes that are targeted by \cite{ATLAS-CONF-2023-001}. Both top quark decay (left) and production (middle and right) CLFV processes are shown.}
 \label{fig:ATLAS_FD}
 \end{center}
\end{figure}

Due to limited statistics, event yields of the \acp{SR} are directly used to interpret the observation, which are shown in Figure~\ref{fig:ATLAS_results2}. An upper limit at 95\% \ac{CL} is placed on the branching fraction of $\mathcal{B}(\textsf{t}\rightarrow\upmu\uptau\textsf{q})$ $<$ 1.1 $\times$ 10$^{-6}$. The corresponding constraint on the \ac{WC} improves the previous bound~\cite{Chala:2018agk} by nearly a factor of 30.

\begin{figure}[tbh!]
 \begin{center}
 \begin{tabular}{cc}
  \includegraphics[width=0.48\textwidth]{figures/Part3/History/ATLAS_results3}&
  \includegraphics[width=0.48\textwidth]{figures/Part3/History/ATLAS_results4}\\
 \end{tabular}
 \caption{The left (right) histogram shows the pre-fit (post-fit) event yields of various regions studied by \cite{ATLAS-CONF-2023-001}.}
 \label{fig:ATLAS_results2}
 \end{center}
\end{figure}
%%%%%%%%%%%%%%%%%%%%%%%%%%%%%%%%%%%%%%%%%%%
%%%%%%%%%%%%%%%%%%%%%%%%%%%%%%%%%%%%%%%%%%%
\section{CMS}
\label{sec:CLFV_CMS}

The \ac{CMS} Collaboration followed up with a search for $\emut{q}$ interactions using data corresponds to 138 fb$^{-1}$~\cite{CMS:2022ztx}. Unlike the previous ATLAS analysis~\cite{ATLAS-CONF-2018-044}, this \ac{CMS} analysis targets final states with two leptons and a hadronically decaying top quark. Both top quark production and decay signals are considered by this analysis. Operators with different Lorentz structures are considered separately. Representative Feynman diagrams are shown in Figure~\ref{fig:CMS_FD}. 

\begin{figure}[tbh!]
 \begin{center}
 \begin{tabular}{ccc}
  \includegraphics[width=0.31\textwidth]{figures/Part3/History/CMS_TT}&
  \includegraphics[width=0.33\textwidth]{figures/Part3/History/CMS_ST1}&
  \includegraphics[width=0.31\textwidth]{figures/Part3/History/CMS_ST2}\\
 \end{tabular}
 \caption{Representative Feynman diagrams for the signal processes that are targeted by~\cite{CMS:2022ztx}. Both top quark decay (left) and production (middle and right) CLFV processes are shown.}
 \label{fig:CMS_FD}
 \end{center}
 \end{figure}
 
A \ac{BDT} using multiple discriminating variables is trained to further enhance the sensitivity. Distributions of the \ac{BDT} discriminator are shown in Figure~\ref{fig:CMS_results}. An upper limit at 95\% \ac{CL} is placed on the branching fraction of $\mathcal{B}(\textsf{t}\rightarrow\upmu\uptau\textsf{q})$ $<$ 7 $\times$ 10$^{-8}$, which improves the previous bound established by the \ac{ATLAS} Collaboration~\cite{ATLAS-CONF-2018-044} by two orders of magnitude.
 
 \begin{figure}[tbh!]
 \begin{center}
 \begin{tabular}{cc}
  \includegraphics[width=0.48\textwidth]{figures/Part3/History/CMS_results1}&
  \includegraphics[width=0.48\textwidth]{figures/Part3/History/CMS_results2}\\
 \end{tabular}
 \caption{The left (right) histogram shows the distribution of the \ac{BDT} discriminator in regions with exactly (more than) one b-tagged jet.}
 \label{fig:CMS_results}
 \end{center}
\end{figure}