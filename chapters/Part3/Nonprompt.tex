\chapter{Nonprompt Background Estimation}
\label{chap:Nonprompt}

In this analysis, the term \emph{prompt} leptons refers to leptons that originate from the \ac{CLFV} vertex, the Drell-Yan process, or an electroweak boson decay, including leptons from $\uptau$ decays if the $\uptau$ lepton originates from the latter two processes. \emph{Nonprompt} leptons refer to leptons that originate from hadron decays and photon conversions, as well as particles misidentified as leptons. Nonprompt leptons are suppressed through isolation requirements and an MVA-based identification specifically trained to reject them.

\emph{Nonprompt} backgrounds are defined to be backgrounds with at least one \emph{nonprompt} lepton passing the \emph{tight} selection criteria, in this case generally dominated by Drell-Yan and $\ttbar$ production. An accurate estimation of \emph{nonprompt} backgrounds is difficult to achieve through MC modelling. Therefore, a data-driven technique called the ``\mm'' \cite{Gillam:2014xua} is used to estimate the \emph{nonprompt} backgrounds. 

A brief description of the \mm~in its simplist form is given in \autoref{sec:MM} followed by its generalization and implementation in \autoref{sec:MR}. This method is validated using three \acp{VR} and is described in \autoref{sec:VR}. Lastly, the \emph{nonprompt} estimation in the \ac{SR} is presented in \autoref{sec:MMSR}.
%%%%%%%%%%%%%%%%%%%%%%%%%%%%%%%%%%%%%%%%%%%%%%%%%%%%%%%%%%%%
%%%%%%%%%%%%%%%%%%%%%%%%%%%%%%%%%%%%%%%%%%%%%%%%%%%%%%%%%%%%

\section{The Matrix Method}
\label{sec:MM}

The \mm~is a data driven technique used to estimate the fraction of \emph{nonprompt} leptons that pass a given lepton selection, referred to as ``\emph{tight}''. The \emph{tight} selection usually incorporates tight lepton identification and isolation requirements and corresponds to full lepton selection used in an analysis. The \emph{loose} selection is obtained by loosening the \emph{tight} selection. The \emph{loose} selection is used as a baseline such that any \emph{loose} leptons fall into one of the two exclusive categories: \emph{tight} or \emph{not tight}. The \mm~deals with  \emph{prompt} and \emph{nonprompt} leptons separately. As a result, \emph{prompt} and \emph{nonprompt} efficiencies are introduced, as is illustrated in Figure~\ref{fig:Matrix_Method}.

 \begin{figure}[tbh!]
 \begin{center}
 \begin{tabular}{c}
 \includegraphics[width=0.8\textwidth]{figures/Part3/Nonprompt/matrix}
 \end{tabular}
 \caption{Illustration of the \emph{prompt} efficiency $r$ and the \emph{nonprompt} efficiency $f$.}
 \label{fig:Matrix_Method}
 \end{center}
\end{figure}

In a simplified scenario with only one lepton in the final state, the \emph{prompt} efficiency $r$ measures the probability of \emph{prompt} leptons pass \emph{tight} selection. It is treated as an observable that can be obtained through measurement,

\begin{equation}
r=\frac{n_P^{T}}{n_P^{T}+n_P^{\overline{T}}},
 \label{eq:real_rate}
\end{equation}
in which $n_P^{T}$/$n_P^{\overline{T}}$ denotes the number of events with a \emph{prompt} lepton that is \emph{tight}/\emph{not tight}.

Similarly, \emph{nonprompt} efficiency $f$ can be expressed as,

\begin{equation}
f=\frac{n_{N}^{T}}{n_{N}^{T}+n_{N}^{\overline{T}}},
 \label{eq:fake_rate}
\end{equation}
in which $n_{N}^{T}$/$n_{N}^{\overline{T}}$ denotes the number of events with a \emph{nonprompt} lepton that is \emph{tight}/\emph{not tight}.

The measurement of $r/f$ is often performed in dedicated control regions, where high purity of \emph{prompt}/\emph{nonprompt} leptons is expected. These regions are referred to as the \ac{MR}. It is assumed that $r/f$ is a universal property of \emph{prompt}/\emph{nonprompt} leptons that is independent of physics processes. Therefore, $r/f$ extracted from \ac{MR} can be used to estimate the contamination of \emph{nonprompt} leptons in a different region (e.g. \ac{SR}) even though these two regions are orthogonal to each other.

In this simplified scenario, the total number of events in the region of interest (e.g. \ac{SR}/\ac{VR}) with a \emph{tight}/\emph{not tight} lepton can be expressed in a system of equations,

\begin{equation}
\begin{split}
N^{T}&=N_{P}^{T}+N_{N}^{T}\\
N^{\overline{T}}&=N_{P}^{\overline{T}}+N_{N}^{\overline{T}},
\end{split}
\end{equation}

in which capital letter ``$N$'' is used to indicate that these numbers are referring to events in a region that is different from \ac{MR}. $N_{P}^{\overline{T}}$/$N_{N}^{\overline{T}}$ can be expressed in terms of $r/f$ and $N_{P}^{T}$/$N_{N}^{T}$ according to Equation \ref{eq:real_rate}/\ref{eq:fake_rate} and the assumption that r/f remains the same across different regions,

\begin{equation}
\begin{split}
N^{T}&=r\frac{N_{P}^{T}}{r}+f\frac{N_{N}^{T}}{f}\\
N^{\overline{T}}&=(1-r)\frac{N_{P}^{T}}{r}+(1-f)\frac{N_{N}^{T}}{f}.
\end{split}
\label{eq:sys_eq}
\end{equation}

Equation~\ref{eq:sys_eq} can also be expressed in the form of matrix,

\begin{equation}
 \begin{pmatrix}
 N^{T}\\
 N^{\overline{T}}
 \end{pmatrix}=\begin{pmatrix}
r&f\\
1-r&1-f
 \end{pmatrix}\begin{pmatrix}
 N_{P}^{T}/r\\
 N_{N}^{T}/f
 \end{pmatrix}.
 \label{eq:matrix}
 \end{equation}
 
Regions that correspond to the two numbers that appear in the righthand side vector of Equation~\ref{eq:matrix} are referred to as the ``\acp{AR}'', which can be constructed using experimental data. The estimation of \emph{nonprompt} background, denoted by $N_{N}^T$, can be obtained by a simple matrix inversion. 
%%%%%%%%%%%%%%%%%%%%%%%%%%%%%%%%%%%%%%%%%%%%%%%%%%%%%%%%%%%%
%%%%%%%%%%%%%%%%%%%%%%%%%%%%%%%%%%%%%%%%%%%%%%%%%%%%%%%%%%%%

\section{Generialization and Implementation of the  Matrix Method}
\label{sec:MR}

The description in previous section deals with a scenario where only one lepton is studied. This analysis uses a generalized version of the \mm, where all three \emph{tight} leptons are considered to be possibly \emph{nonprompt}. Equation~\ref{eq:matrix} is generalized as,

\begin{equation}
\hspace{-3.7em}
 \resizebox{1.2\linewidth}{!}{%
 $
 \begin{pmatrix}
 N^{TTT}\\\\
 N^{TT\overline{T}}\\\\
 N^{T\overline{T}T}\\\\
 N^{T\overline{T}\overline{T}}\\\\
 N^{\overline{T}TT}\\\\
 N^{\overline{T}T\overline{T}}\\\\
 N^{\overline{T}\overline{T}T}\\\\
 N^{\overline{T}\overline{T}\overline{T}}
 \end{pmatrix}=\begin{pmatrix}
r_1r_2r_3&r_1r_2f_3&r_1f_2r_3&r_1f_2f_3&f_1r_2r_3&f_1r_2f_3&f_1f_2r_3&f_1f_2f_3\\\\
r_1r_2(1-r_3)&r_1r_2(1-f_3)&r_1f_2(1-r_3)&r_1f_2(1-f_3)&f_1r_2(1-r_3)&f_1r_2(1-f_3)&f_1f_2(1-r_3)&f_1f_2(1-f_3)\\\\
r_1(1-r_2)r_3&r_1(1-r_2)f_3&r_1(1-f_2)r_3&r_1(1-f_2)f_3&f_1(1-r_2)r_3&f_1(1-r_2)f_3&f_1(1-f_2)r_3&f_1(1-f_2)f_3\\\\
r_1(1-r_2)(1-r_3)&r_1(1-r_2)(1-f_3)&r_1(1-f_2)(1-r_3)&r_1(1-f_2)(1-f_3)&f_1(1-r_2)(1-r_3)&f_1(1-r_2)(1-f_3)&f_1(1-f_2)(1-r_3)&f_1(1-f_2)(1-f_3)\\\\
(1-r_1)r_2r_3&(1-r_1)r_2f_3&(1-r_1)f_2r_3&(1-r_1)f_2f_3&(1-f_1)r_2r_3&(1-f_1)r_2f_3&(1-f_1)f_2r_3&(1-f_1)f_2f_3\\\\
(1-r_1)r_2(1-r_3)&(1-r_1)r_2(1-f_3)&(1-r_1)f_2(1-r_3)&(1-r_1)f_2(1-f_30&(1-f_1)r_2(1-r_3)&(1-f_1)r_2(1-f_3)&(1-f_1)f_2(1-r_3)&(1-f_1)f_2(1-f_3)\\\\
(1-r_1)(1-r_2)r_3&(1-r_1)(1-r_2)f_3&(1-r_1)(1-f_2)r_3&(1-r_1)(1-f_2)f_3&(1-f_1)(1-r_2)r_3&(1-f_1)(1-r_2)f_3&(1-f_1)(1-f_2)r_3&(1-f_1)(1-f_2)f_3\\\\
(1-r_1)(1-r_2)(1-r_3)&(1-r_1)(1-r_2)(1-f_3)&(1-r_1)(1-f_2)(1-r_3)&(1-r_1)(1-f_2)(1-f_3)&(1-f_1)(1-r_2)(1-r_3)&(1-f_1)(1-r_2)(1-f_3)&(1-f_1)(1-f_2)(1-r_3)&(1-f_1)(1-f_2)(1-f_3)
 \end{pmatrix}\begin{pmatrix}
 N_{PPP}^{TTT}/r_1r_2r_3\\\\
 N_{PPN}^{TTT}/r_1r_2f_3\\\\
 N_{PNP}^{TTT}/r_1f_2r_3\\\\  
 N_{PNN}^{TTT}/r_1f_2f_3\\\\
 N_{NPP}^{TTT}/f_1r_2r_3\\\\
 N_{NPN}^{TTT}/f_1r_2f_3\\\\
 N_{NNP}^{TTT}/f_1f_2r_3\\\\
 N_{NNN}^{TTT}/f_1f_2f_3
 \end{pmatrix}.
 $}
 \label{eq:matrix_method3}
 \end{equation}
 
All but the first number that appear in the righthand side vector correspond to events with at least one \emph{nonprompt} lepton that pass \emph{tight} selection criteria. Therefore, the overall \emph{nonprompt} background is expresses as,
 
\begin{equation}
N_{Nonprompt}^{TTT} = N_{PPN}^{TTT} + N_{PNP}^{TTT} + N_{PNN}^{TTT} + N_{NPP}^{TTT} + N_{NPN}^{TTT} + N_{NNP}^{TTT} + N_{NNN}^{TTT},
\end{equation}
 
which can be obtained by constructing 8 \acp{AR} to form that lefthand side vector. Secondly, the 8 $\times$ 8 matrix is constructed and inverted. Lastly, the righthand side vector can be obtained by multiplying the lefthand side vector to the inverted matrix.

Only two \acp{PD} ``SingleElectron'' and ``SingleMuon'' are used in the construction of \ac{MR} in 2016 and 2017 while ``SingleElectron'' is replaced with ``EGamma'' in 2018. In addition to \acp{PD}, the measurements of $r$/$f$ also utilize the $\ttbar$ sample and all \ac{MC} samples listed under the ``\emph{prompt} background'' category in Table~\ref{tab:MCsample}. Depending on the flavor of the leading lepton in \ac{MC}, events are selected with either a single-electron or a single-muon trigger, which is summarized in Table~\ref{tab:RandF_trigger}. Data events are selected with the same \ac{HLT} triggers as well but events in ``SingleMuon'' (``SingleElectron'' or ``EGamma'') \ac{PD} are only accepted if the leading lepton is a muon (electron).

\begin{table}[th]
\sffamily
\centering
\begin{tabular}{llllll}
\toprule
Channel   & Path       & Dataset  & 2016 & 2017 & 2018 \\ \midrule
\multirow{2}{*}{Electron} & HLT\_Ele27\_WPTight\_Gsf  & Data \& MC & \checkmark & - & - \\ 
           & HLT\_Ele35\_WPTight\_Gsf & Data \& MC & - & \checkmark & \checkmark \\ \hline
\multirow{1}{*}{Muon}  & HLT\_IsoMu27 & Data \& MC & \checkmark & \checkmark & \checkmark \\ \bottomrule
\end{tabular}
\caption{Summary of the \ac{HLT} triggers used in the measurement of $r$ and $f$.}
\label{tab:RandF_trigger}
\end{table}

Both $r$ and $f$ are parameterized in bins of lepton $\pt$, $|\eta|$, and jet multiplicity. The bin range is optimized to retain sufficient statistics for each bin:

\begin{itemize}
\item Electron $\pt$ bin range: \{20.0, 24.6, 28.8, 33.0, 37.2, 41.4, 46.1, 52.1, 59.3, 68.3, 82.7, 110.6\} GeV,
\item Muon $\pt$ bin range: \{20.0, 23.8, 27.7, 31.3, 35.0, 38.9, 42.8, 45.6, 50.7, 59.5, 72.9, 94.3\} GeV,
\item $|\eta|$ bin range: \{0, 0.8, 1.6, 2.4\},
\item Jet multiplicity: \{0 jet, $\geq$ 1 jet\}.
\end{itemize}

The jet multiplicity bin is a proxy for variation of the composition of physics processes. In addition to requiring at least one jet, the actual \ac{MR} corresponds to the second jet multiplicity bin requires no more than one b-tagged jet as this is also required in the \ac{SR}.

The \emph{nonprompt} efficiency is measured in same-sign dilepton regions, in which the leading lepton in $\pt$, used as a \emph{tag}, is required to be matched with trigger objects within $\mathrm{\Delta}R~<$ 0.2. The sub-leading lepton is required to pass the \emph{loose} selection and is taken as the \emph{probe}. Events that have two same-sign electrons with an invariant mass between 76 GeV and 106 GeV are removed from \ac{MR} to suppress the backgrounds that originate from charge misidentification. No such requirement has been introduced to the muon \ac{MR} due to its negligible rate of  charge misidentification.

The contribution from prompt backgrounds, estimated from \ac{MC} simulation, are subtracted from data. A representative composition of backgrounds in \ac{MR} is shown in Figure~\ref{fig:MRexample}. 

\begin{figure}[tbh!]
 \begin{center}
 \begin{tabular}{cc}
 \includegraphics[width=0.45\textwidth]{figures/Part3/Nonprompt/MR/FlepPt}&
 \includegraphics[width=0.45\textwidth]{figures/Part3/Nonprompt/MR/TlepPt} \\
 \end{tabular}
 \caption{Distribution of lepton $\pt$ in a representative electron \emph{nonprompt} efficiency \ac{MR}. In this particular example, both ee and $\upmu$e flavor composites are considered. At least one jet and at most one b-tagged jet are required (the second jet multiplicity bin). \emph{Probe} electron is required to have $1.6<|\eta|<2.4$ (the third $\eta$ bin). Contamination from \emph{prompt} backgrounds are estimated with \ac{MC} simulation, and are shown as histograms. The data are shown as filled points. From left to right: \emph{loose} (i.e. \emph{tight + not tight}) electron $\pt$, \emph{tight} electron $\pt$.}
 \label{fig:MRexample}
 \end{center}
\end{figure}

The fake efficiency $f$ is calculated as:

\begin{equation}
f=\frac{n_{data}^{tag+tight}-n_{MC(prompt)}^{tag+tight}}{n_{data}^{tag+loose}-n_{MC(prompt)}^{tag+loose}},
\label{eq:f_eq}
\end{equation}  

where the numerator is selected with one \emph{tag} and one \emph{tight} lepton while the denominator is selected with one \emph{tag} and one \emph{loose} lepton. The selection criteria for \emph{tag}, \emph{loose}, and \emph{tight} lepton is summarised in Table~\ref{tab:MR}.

\begin{table}[th]
\sffamily
\centering
\begin{tabular}{ccccc}
\toprule
Lepton   &Selection   & \emph{loose} & \emph{tag} & \emph{tight}$^{\textsf{ii}}$\\ \midrule
\multirow{4}{*}{Electron} & $p_{T}$  & $>$ 20 GeV & $>$ 38 GeV$^{\textsf{i}}$ & $>$ 20 GeV \\  
     & $I_{\textsf{mini}}^{\textsf{rel}}$  & $<$0.4 & $<$0.1 & $<$0.12 \\
     & \TOP   & $>$-0.9   & $>$0.95 & $>$0.9 \\ 
     & Match with trigger objects   & - & \checkmark & -  \\ \midrule
\multirow{5}{*}{Muon} & $p_{T}$ & $>$ 20 GeV & $>$ 30 GeV & $>$ 20 GeV \\
     & $I_{\textsf{mini}}^{\textsf{rel}}$  & $<$0.4 & $<$0.1 & $<$0.12 \\
     & Cut-based ID  & - & Medium WP & Medium WP \\
     & \TOP   & $>$0.5 & $>$0.9 & $>$0.9 \\ 
     & Match with trigger objects   & - & \checkmark & -  \\ \bottomrule
\end{tabular}
\caption{Summary of the lepton selections needed for the measurement of $r$ and $f$. Please note: (i) the minimum $p_{T}$ cut for \emph{tag} electron in 2016 dataset is reduced to 30 GeV to adjust for the trigger threshold, and (ii) the \emph{tight} selection here is the same as the \emph{tight} lepton selection described in \autoref{sec:Leptons}.}
\label{tab:looseandtight}
\end{table}

\begin{figure}[tbh!]
 \begin{center}
 \begin{tabular}{c}
 \includegraphics[width=0.85\textwidth]{figures/Part3/Nonprompt/MR/fake_eff}
 \end{tabular}
 \caption{Representative \emph{nonprompt} electron efficiency measured in data events. These plots correspond to the first $|\eta|$ bin ($|\eta|<$0.8) and the second jet multiplicity bin. Events selected Error bars displayed in these plots include statistical uncertainty only. From left to right: electron $f$, muon $f$.}
 \label{fig:fake_eff}
 \end{center}
\end{figure}

The measured \emph{nonprompt} efficiency $f$ exhibits a dependency on flavor composition, as is shown in Figure~\ref{fig:fake_eff}. This dependency is treated as a source of the systematic uncertainties of the nonprompt estimation and is further discussed in \autoref{sec:NonUnc}.

The \emph{prompt} efficiency $r$ is estimated in simulated $\ttbar$ events in opposite-sign dilepton regions. The same lepton selection listed in Table~\ref{tab:looseandtight} is used to perform the \emph{Tag-and-Probe}. The leading lepton in $p_{T}$ is used as a \emph{tag} while the oppositely charged sub-leading lepton is taken as a \emph{probe}. The variation of $r$ between different flavor composition is negligible, as is shown in Figure~\ref{fig:real_eff}. Therefore, only e$\upmu$ events are used to measure \emph{prompt} efficiency in order to minimise the contamination of \emph{nonprompt} leptons.

\begin{figure}[tbh!]
 \begin{center}
 \begin{tabular}{c}
 \includegraphics[width=0.85\textwidth]{figures/Part3/Nonprompt/MR/real_eff}
 \end{tabular}
 \caption{Representative \emph{prompt} efficiency measured in simulated $t\bar{t}$ events. These plots correspond to the first $|\eta|$ bin ($|\eta|<$0.8) and the second jet multiplicity bin. Error bars displayed in these plots include statistical uncertainty only. From left to right: electron $r$, muon $r$.}
 \label{fig:real_eff}
 \end{center}
\end{figure}

The selection criteria for various \acp{MR} is summarised in Table~\ref{tab:MR}.

\begin{table}[th]
\sffamily
\centering
\resizebox{\textwidth}{!}{ 
\begin{tabular}{cccccccc}
\toprule
\multirow{2}{*}{Observable}   & \multirow{2}{*}{jet bin}          &  \multirow{2}{*}{\vtop{\hbox{\strut $\#$ of selected}\hbox{\strut~~~leptons}}} &   \multirow{2}{*}{\vtop{\hbox{\strut lepton flavor}\hbox{\strut~composite}}} & \multirow{2}{*}{$|\sum_iC_i|$} & \multirow{2}{*}{OffZ}               &  \multirow{2}{*}{njet}  & \multirow{2}{*}{nbjet}\\ 
   &  &  &   &  &                &    & \\ \midrule
\multirow{2}{*}{$f$}     & 0 jet                & 2                                & any                                   & 2                       & same-sign ee   & = 0                    & = 0   \\  
                                   & 1 or more jet  & 2                                &  any                                  & 2                         &  same-sign ee    & $\geq$ 1 & $\leq$ 1\\ \midrule
\multirow{2}{*}{$r$}    & 0 jet                & 2                                 & e$\upmu$ only                     & 0                        & -    & = 0                      & = 0     \\  
                                 & 1 or more jet  & 2                                  & e$\upmu$ only                      & 0                        & -      & $\geq$ 1 & $\leq$ 1 \\ \bottomrule  
\end{tabular}
}
\caption{Summary of the cuts applied to the $r$/$f$ measurement region.}
\label{tab:MR}
\end{table}
%%%%%%%%%%%%%%%%%%%%%%%%%%%%%%%%%%%%%%%%%%%%%%%%%%%%%%%%%%%%
%%%%%%%%%%%%%%%%%%%%%%%%%%%%%%%%%%%%%%%%%%%%%%%%%%%%%%%%%%%%

\section{Validation of the Matrix Method}
\label{sec:MMVR}

The performance of the \mm~is validated using three regions that are tangential to the \ac{SR}, referred to as \acp{VR}. In these \acp{VR}, \emph{prompt} backgrounds are estimated using MC simulation while \emph{nonprompt} background is estimated with the \mm. A summary of the selections applied to these VRs is given in \autoref{chap:Selection}. 

Distribution of the leading lepton $\eta$ and jet multiplicity are shown in Figure~\ref{fig:VR_matrix_eee}-\ref{fig:VR_matrix_mumumu}. Good agreement between data and background estimate has been observed in all three \acp{VR}.

\begin{figure}[tbh!]
 \begin{center}
 \begin{tabular}{cc}
 \includegraphics[width=0.45\textwidth]{figures/Part3/Nonprompt/VR/eee/lep1Eta}&
 \includegraphics[width=0.45\textwidth]{figures/Part3/Nonprompt/VR/eee/njet} \\
 \end{tabular}
 \caption{Distributions of the leading lepton $\eta$ (left) and the jet multiplicity (right) in the eee \emph{nonprompt} VR. The data are shown as filled points and the SM background predictions as histograms. The nonprompt background is estimated using control samples in data, while other backgrounds are estimated using MC simulation. The hatched bands indicate statistical and systematic uncertainties for the SM background predictions. The last bin of the right histogram includes the overflow events.}
 \label{fig:VR_matrix_eee}
 \end{center}
\end{figure}

\begin{figure}[tbh!]
 \begin{center}
 \begin{tabular}{cc}
  \includegraphics[width=0.45\textwidth]{figures/Part3/Nonprompt/VR/emul/lep1Eta}&
 \includegraphics[width=0.45\textwidth]{figures/Part3/Nonprompt/VR/emul/njet} \\
 \end{tabular}
 \caption{Distributions of the leading lepton $\eta$ (left) and the jet multiplicity (right) in the $\emul$ \emph{nonprompt} VR. The data are shown as filled points and the SM background predictions as histograms. The nonprompt background is estimated using control samples in data, while other backgrounds are estimated using MC simulation. The hatched bands indicate statistical and systematic uncertainties for the SM background predictions. The last bin of the right histogram includes the overflow events.}
 \label{fig:VR_matrix_emul}
 \end{center}
\end{figure}

\begin{figure}[tbh!]
 \begin{center}
 \begin{tabular}{cc}
   \includegraphics[width=0.45\textwidth]{figures/Part3/Nonprompt/VR/mumumu/lep1Eta}&
 \includegraphics[width=0.45\textwidth]{figures/Part3/Nonprompt/VR/mumumu/njet} \\
 \end{tabular}
 \caption{Distributions of the leading lepton $\eta$ (left) and the jet multiplicity (right) in the $\mmm$ \emph{nonprompt} VR. The data are shown as filled points and the SM background predictions as histograms. The nonprompt background is estimated using control samples in data, while other backgrounds are estimated using MC simulation. The hatched bands indicate statistical and systematic uncertainties for the SM background predictions. The last bin of the right histogram includes the overflow events.}
 \label{fig:VR_matrix_mumumu}
 \end{center}
\end{figure}
%%%%%%%%%%%%%%%%%%%%%%%%%%%%%%%%%%%%%%%%%%%%%%%%%%%%%%%%%%%%
%%%%%%%%%%%%%%%%%%%%%%%%%%%%%%%%%%%%%%%%%%%%%%%%%%%%%%%%%%%%

\section{Nonprompt estimate in SR}
\label{sec:MMSR}

The \mm~is used to estimate \emph{nonprompt} background in the \ac{SR}. Distributions of the LFV-e$\upmu$ mass and the Z boson mass are shown in Figure~\ref{fig:SR_DataDriven_1}. When compared to background estimate from pure MC simulation (Figure~\ref{fig:SR}), the updated background template is smoother with lower statistical uncertainties. 

\begin{figure}[tbh!]
 \begin{center}
 \begin{tabular}{cc}
  \includegraphics[width=0.45\textwidth]{figures/Part3/Nonprompt/SR/Memu}&
 \includegraphics[width=0.45\textwidth]{figures/Part3/Nonprompt/SR/Zmass} \\
 \end{tabular}
 \caption{Distributions of the LFV-e$\upmu$ mass (left) and the Z boson mass (right) in \ac{SR}. The data are shown as filled points and the \ac{SM} background predictions as histograms. The \emph{nonprompt} background is estimated using control samples in data, while other backgrounds are estimated using \ac{MC} simulation. The hatched bands indicate statistical and systematic uncertainties for the \ac{SM} background predictions. The normalisation of the signal processes is chosen arbitrarily for improved visualisation. The last bin of both histograms includes the overflow events.}
 \label{fig:SR_DataDriven_1}
 \end{center}
\end{figure}

The number of expected events from various kinds of backgrounds are shown in Table \ref{tab:eventcount}. 

\begin{table}[th]
\sffamily
\centering
\caption{Expected background contributions and the number of events observed in data collected during 2016--2018. The statistical and systematic uncertainties are added in quadrature. The category ``other backgrounds'' includes smaller background contributions containing one or two top quarks plus a boson or quark. The \ac{CLFV} signal, generated with $\WC{\textsf{vector}}{\emut{u}}/\Lam^2=1\TeV^{-2}$, is also listed for reference. The signal yields include contributions from both top production and decay modes.}
\begin{tabular}{ccc}
\toprule
Process & m(e$\upmu$)<150 GeV & m(e$\upmu$)>150 GeV \\
\noalign{\vskip 1mm}
\midrule
\noalign{\vskip 1mm}
Nonprompt & $351\pm92$ & $146\pm38$\\
WZ & $275\pm64$ & $145\pm35$\\
ZZ & $33.2\pm6.5$ & $13.1\pm2.6$\\
VVV & $17.0\pm8.5$ & $12.0\pm6.0$\\
$\ttbar$W & $47.6\pm10.0$ & $40.0\pm9.1$\\
$\ttbar$Z & $39.1\pm7.9$ & $25.8\pm5.4$\\
$\ttbar$H & $28.2\pm4.5$ & $10.0\pm1.6$\\
tZq & $5.5\pm1.1$ & $2.5\pm0.5$\\
Other & $7.3\pm3.7$ & $4.5\pm2.3$\\
Total expected & $805\pm123$ & $398\pm57$\\
\noalign{\vskip 1mm}
Data & 783 & 378\\
\noalign{\vskip 1mm}
\midrule
\noalign{\vskip 1mm}
CLFV & $207\pm15$ & $4440\pm215$\\
\noalign{\vskip 1mm}
\bottomrule
\end{tabular}
\label{tab:eventcount}
\end{table}

