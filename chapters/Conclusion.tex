\addtocontents{toc}{\vspace{6pt}}% Dirty hack to increase spacing
\chapter{Summary and Conclusions}
\label{chap:Conclusion}

This thesis presents results from a search for charged-lepton flavor violation in both top quark production and decay processes. The data used were collected by the \ac{CMS} experiment during 2016--2018 and correspond to an integrated luminosity of 138 fb$^{-1}$. Events were selected for analysis if they contained exactly three charged leptons--one electron and one muon of opposite electric charge as well as one additional electron or muon. Events must also contain at least one jet of which no more than one is associated with a bottom quark. An effective field theory approach is used for parametrizing the charged-lepton flavor-violating interactions. Boosted decision trees are used to distinguish a possible signal from the background. No significant excess is observed over the prediction from the standard model. Upper limits at the 95\% confidence level are set on the branching fractions involving up (charm) quarks, $\tto{u}$ ($\tto{c}$), of $0.032 (0.498) \times 10^{-6}$, $0.022 (0.369) \times 10^{-6}$, and $0.012 (0.216) \times 10^{-6}$ for tensor, vector, and scalar interactions, respectively. These limits constitute the most stringent ones to date on these processes, improving the existing limits by roughly one order of magnitude.

Additionally, the e$\uptau$tq and $\upmu\uptau$tq interactions are also being studied currently using only simulated events in a second search. The tau leptons considered in this search undergo hadronic decay and are selected with a \ac{NN}-based (hadronic tau) identification algorithm. Selected events in this search contain exactly one hadronic tau and two charged leptons (electron or muon) and are divided into 18 search bins. This search is expected to place competitive upper limits on branching fractions of t$\rightarrow$e$\uptau$q and t$\rightarrow\upmu\uptau$q processes while further improving the upper limits established by the first analysis on $\tto{q}$ branching fractions. 

So far, all the \ac{CLFV} searches performed by the \ac{CMS} Collaboration report excellent agreement with the \ac{SM} expectations. Many of these searches, including the two presented in this analysis, are limited by statistics, where a significant improvement in sensitivity is expected upon the arrival of the \ac{HL-LHC} dataset. To cope with the harsh environment of the \ac{HL-LHC}, the \ac{CMS} Experiment is planning to incorporate the tracking information at the Level-1 trigger. This will provide a much-needed handle to reduce data value and mitigate the effects of pile-up interactions. The Level-1 track-finding algorithm along with one of its applications to electron triggers are also presented in this thesis.