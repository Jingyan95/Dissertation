\chapter{Introduction}
\label{chap:Introduction}

The 20th century was not only dominated by unprecedented geopolitical events, but it also witnessed a giant leap in our understanding of nature. Among many great scientific achievements, a remarkable insight into the fundamental structure of matter was developed through the work of thousands of physicists worldwide, spanning several decades. At the most fundamental scale, everything in our universe is made up of indivisible building blocks known as fundamental particles. These point-like fundamental particles manifest in two kinds: fermions and bosons. Fermions comprise all the matter in our universe while bosons provide mechanisms to explain fundamental forces and and the origin of particle mass. Our best understandings of all the known fundamental particles and three of the four known fundamental forces (electromagnetism, weak, and strong interactions -- excluding gravity) are encoded in the theory known as the Standard Model (\ac{SM}) of particle physics.

The \ac{SM} is a quantum field theory that obeys the principle of gauge invariance in which the dynamics of the full system are invariant under local gauge transformations. It comprises an \ac{EW} sector and a strong sector, where three fundamental forces emerge naturally. The \ac{EW} sector provides a unified description of the electromagnetism and the weak interaction under the \ew~gauge symmetry while the strong sector features the theory of Quantum Chromodynamics (\ac{QCD}), which describes the strong interaction with the $SU(3)_{C}$ Lie group. Since its completion in the 1970s, the \ac{SM} has been immensely successful in describing all the known fundamental particles in a self-consistent way. The properties of these fundamental particles predicted by the \ac{SM} agree very well with the vast majority of the experimental data. However, the \ac{SM} falls short of being the most fundamental description of nature as it does not incorporate any theory of gravity, provide any viable candidates for dark matter, or fully explain matter-antimatter asymmetry. Consequently, it is widely assumed that the \ac{SM} is only one aspect of a more fundamental theory that we have yet to uncover.

One of the most mysterious aspects of the \ac{SM} is the concept of fermion flavor, that is, fermions exist in three generations. With the exception of the masses, fermions of different generations can be viewed as identical copies of each other. The majority of the free parameters of the \ac{SM}, such as the masses of fermions, belong to the flavor sector, and the \ac{SM} remains completely silent on why these parameters are tuned to the values observed in experiments. A subset of these free parameters that have yet to be measured precisely is related to the flavor mixings. When participating in the weak interactions, different-flavor fermions are characterized by distinct quantum states known as the flavor eigenstates, which might be different from the eigenstates of the free Hamiltonian (i.e. mass eigenstates). The mismatch of quark mass eigenstates and flavor eigenstates may occur, and it leads to flavor-changing interactions via the exchange of W bosons. In other words, quark flavor is not conserved in the \ac{SM}, and the mixing of quark flavors is characterized by free parameters that can only be obtained through experimental measurements. 

Unlike quark flavor, lepton flavor is conserved in the \ac{SM} with massless neutrinos. However, the observation of neutrino oscillations~\cite{Super-Kamiokande:1998kpq,SNO:2002tuh} confirms the existence and the mixing of massive neutrinos, and it also indicates that lepton flavor violation (LFV), meaning local interaction that alters lepton flavor, is expected to occur in the charged-lepton sector. The charged-lepton flavor violation (\ac{CLFV}) processes can be divided into two main categories depending on the energy scale of the interactions. The \ac{CLFV} processes that only involve light particles, such as $\upmu\rightarrow\textsf{e}\gamma$ and $\upmu\rightarrow$eee, generally carry low momentum transfers, thus extremely high sensitivity can be achieved at small but dedicated experiments~\cite{MEGII:2018kmf,Mu3e:2020gyw}. New physics can also manifest in \ac{CLFV} processes involving heavy particles, such as the Z boson, Higgs boson, or top quark (t). In this case, the Large Hadron Collider (\ac{LHC})~\cite{Evans:2008zzb} built by the European Organization for Nuclear Research (\ac{CERN}) could provide the highest sensitivity~\cite{Davidson:2012wn} as it is currently the only machine that is capable of producing all of them.

This thesis describes the research work I did in 2019-2023 on the Compact Muon Solenoid (\ac{CMS}) experiment, including a brief description of the surrounding context and background knowledge. This thesis is organized into four parts, and \autoref{Part1} introduces the theoretical foundation of high energy particle physics, including the electroweak theory, \ac{QCD}, and theories \ac{BSM}. Descriptions of the \ac{LHC} and {CMS} experiment, including the operational and upgrade work that I made direct contributions to, are given in \autoref{Part2}. \autoref{Part3} describes a search for flavor-violating $\emut{q}$ interactions using data collected by the \ac{CMS} detector in 2016-2018. The \ac{CLFV} processes in the production and decay of top quarks are both considered in this search. The scope of this search is expanded to include $\textsf{e}\uptau\textsf{qt}$ and $\upmu\uptau\textsf{qt}$ interactions in a second search, which is described in \autoref{Part4}.




