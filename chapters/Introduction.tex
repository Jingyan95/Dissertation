\chapter{Introduction}
\label{chap:Introduction}

The \ac{SM} of particle physics 

Although the \ac{SM} is considered to be a self-consistent theory

One of the most intriguing aspects of the \ac{SM} is the presence of three copies of each fermions.  

Unlike quark flavor, lepton flavor is conserved in the \ac{SM} with massless neutrinos. However, the observation of neutrino oscillations~\cite{Super-Kamiokande:1998kpq,SNO:2002tuh} confirms the existence and the mixing of massive neutrinos, and it also indicates that lepton flavor violation (LFV) is expected to occur in the charged-lepton sector. The \ac{CERN} \ac{LHC} can provide sensitivity~\cite{Davidson:2012wn} to \ac{CLFV} searches, including in two- or three-body decays of heavy particles, X$\rightarrow\ell\ell^{\prime}$(Y), and in heavy-particle production, pp$ \rightarrow\ell\ell^{\prime}$(X). Here, X refers to a heavy \ac{SM} particle such as a top quark (t) or a Higgs, W, or Z boson, $\ell(\ell^{\prime}$) are charged leptons with different flavor and opposite charge, Y denotes an additional generic \ac{SM} particle, and p stands for proton. For \ac{CLFV} processes involving the heaviest of all elementary particles, the top quark, competitive sensitivity is predicted at the \ac{LHC} compared to previous bounds on such interactions~\cite{Davidson:2015zza}. Recent flavor anomalies in decays of B mesons, including the recent potential lepton flavor universality violation reported by the \ac{LHCb} experiment~\cite{LHCb:2023zxo}, has prompted renewed experimental interest in \ac{CLFV} searches. Moreover, some models that accommodate these flavor anomalies also suggest that \ac{CLFV} involving a top quark is within the reach of the \ac{LHC} sensitivity~\cite{Kim:2018oih}.

This thesis describes the research work I did in 2019-2023 on the \ac{CMS} experiment, including a brief description of the surrounding context and background knowledge. This thesis is organized into four parts, and the \autoref{Part1} introduces the theoretical foundation of high energy particle physics, including the electroweak theory, \ac{QCD}, and theories \ac{BSM}. Descriptions of the \ac{LHC} and {CMS} experiment, including the operational and upgrade work that I made direct contributions to, are given in the \autoref{Part2}. The \autoref{Part3} describes a search for flavor-violating $\emut{q}$ interactions using data collected by the \ac{CMS} detector in 2016-2018. The scope of this search is expanded to include $\textsf{e}\uptau\textsf{qt}$ and $\upmu\uptau\textsf{qt}$ interactions in a second search, which is described in the \autoref{Part4}.




