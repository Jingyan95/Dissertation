\chapter{Introduction}
\label{chap:Introduction}

The 20th century was not only dominated by unprecedented geopolitical events, it also witnessed a giant leap in our understanding of nature. Among many great scientific achievements, a remarkable insight into the fundamental structure of matter was developed through the works of thousands of physicists, spanning over several decades. At the most fundamental scales, everything in our universe is made up of indivisible building blocks known as fundamental particles. These point-like fundamental particles come with two main kinds: fermions and bosons. Fermions comprise all the matter in our universe while bosons provide mechanisms to explain fundamental forces and and the origin of particle mass. Our best understandings of all the known fundamental particles and and three of the four known fundamental forces are encoded in the theory known as the \ac{SM} of particle physics.

One of the most intriguing aspects of the \ac{SM} is the concept of fermion flavor, that is, fermions exist in three generations. With the exception of the mass and flavor quantum number, fermions of different generations are viewed as identical copies of each other. When participating in the weak interactions, fermions are described by special quantum states known as the flavor eigenstates, which might be different from the eigenstates of free Hamiltonian (i.e. mass eigenstates). The miss match of quark mass eigenstates and flavor eigenstates in the \ac{SM} allows for \ac{FCCC}. In other words, quark flavor is not conserved in the weak interactions.

Unlike quark flavor, lepton flavor is conserved in the \ac{SM} with massless neutrinos. However, the observation of neutrino oscillations~\cite{Super-Kamiokande:1998kpq,SNO:2002tuh} confirms the existence and the mixing of massive neutrinos, and it also indicates that lepton flavor violation (LFV) is expected to occur in the charged-lepton sector. The \ac{CERN} \ac{LHC} can provide sensitivity~\cite{Davidson:2012wn} to \ac{CLFV} searches, including in two- or three-body decays of heavy particles, X$\rightarrow\ell\ell^{\prime}$(Y), and in heavy-particle production, pp$ \rightarrow\ell\ell^{\prime}$(X). Here, X refers to a heavy \ac{SM} particle such as a top quark (t) or a Higgs, W, or Z boson, $\ell(\ell^{\prime}$) are charged leptons with different flavor and opposite charge, Y denotes an additional generic \ac{SM} particle, and p stands for proton. 

For \ac{CLFV} processes involving the heaviest of all elementary particles, the top quark, competitive sensitivity is predicted at the \ac{LHC} compared to previous bounds on such interactions~\cite{Davidson:2015zza}. Recent flavor anomalies in decays of B mesons, including the recent potential lepton flavor universality violation reported by the \ac{LHCb} experiment~\cite{LHCb:2023zxo}, has prompted renewed experimental interest in \ac{CLFV} searches. Moreover, some models that accommodate these flavor anomalies also suggest that \ac{CLFV} involving a top quark is within the reach of the \ac{LHC} sensitivity~\cite{Kim:2018oih}. Therefore, a search for \ac{CLFV} in the top quark sector at the \ac{LHC} could shed light on these flavor anomalies.

This thesis describes the research work I did in 2019-2023 on the \ac{CMS} experiment, including a brief description of the surrounding context and background knowledge. This thesis is organized into four parts, and the \autoref{Part1} introduces the theoretical foundation of high energy particle physics, including the electroweak theory, \ac{QCD}, and theories \ac{BSM}. Descriptions of the \ac{LHC} and {CMS} experiment, including the operational and upgrade work that I made direct contributions to, are given in the \autoref{Part2}. The \autoref{Part3} describes a search for flavor-violating $\emut{q}$ interactions using data collected by the \ac{CMS} detector in 2016-2018. The scope of this search is expanded to include $\textsf{e}\uptau\textsf{qt}$ and $\upmu\uptau\textsf{qt}$ interactions in a second search, which is described in the \autoref{Part4}.




